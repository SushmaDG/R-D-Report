%!TEX root = ../DevaramaniS-[RnD-MT]Report.tex

\begin{document}
	

\chapter{Introduction}
\label{chap:Intro}
%what is the project about
Widespread programming support to carry out simpler robot tasks such as pick and place, navigation and positioning tasks specified in structured environments, is already available~\cite{de2007constraint}. However, the expanding robot applications in various fields has increased the complexity of tasks. Examples of such tasks include navigation in an unstructured environment, 3D manipulation, etc. Researchers have continuously focused on providing programming support to specify and fulfill complex tasks. Some of the software frameworks that were introduced to control and execute desired robot motions, aimed at providing a standardized way to specify such complex tasks. However, these frameworks are currently applied only to the field of manipulators. 

%The robot manipulator tasks description provides the described motions or acceleration to achieve in order to fulfill a given task. By considering the dynamics of the robot model, it is able to execute the expected motions.

The robot manipulators perform repetitive tasks such as pick and place, painting, etc. Due to its constant physical interaction with the environment, the task requirements impose constraints on robot motions. For instance, consider a pick and place scenario, where the arm has to grasp a fragile glass and place it on a workbench. The robot arm is required to execute compliant motion. That is, the end-effector has to grip with a specific force such that the glass neither breaks nor slips out. Additionally, the task might impose multiple constraints, such as the end-effector must place the object perpendicular to the plane by applying limited forces. The arm must satisfy these dynamic constraints, while executing the control commands. Therefore, to execute constrained motion, the \textit{compliance-based} or \textit{task-based} approaches were introduced, where these methods are controlled at force/acceleration level~\cite{han2017interaction}. Unlike the manipulators, mobile robots do not explicitly involve in physical interactions with environment. However, in some circumstances, the objectives demand force/acceleration constraints, when obliged to come in contact with the environment. The current robot navigation approaches follow velocity-based scheme, where the motion is controlled at velocity-level and hence the robot cannot model the constraints in force/acceleration level. 

%The goal of the project is to extend and apply Popov-Vereshchagin solver to mobile robots.


\section{Motivation}
%Why is it important
Currently available approaches to implement simple navigation tasks, where robot moves from one point to another, have successfully been implemented. The focus is mostly on the safe-navigation. Multiple approaches were proposed to address unexpected collision avoidance problem. However, all these approaches follow velocity-controlled method and cannot instantly react to the collisions. Hence, an approach is required to implement \textit{safe motion} of the mobile base. 

\paragraph{}While task specification frameworks have been applied extensively in the field of manipulators, their implementation in the field of mobile robots have always been limited. The main reasons why such frameworks must be utilized in mobile robots are,
\begin{itemize}
	\item The specified tasks are modeled as a composition of constraints. For instance, a navigation task has a force constraint when it comes in contact with an object. Such task can be easily specified using task specification frameworks.
	\item They provide a ``declarative'' way of specifying tasks. For example, if a robot is required to go to a library, instead of commanding ``apply 5Nm torque to left and wheel'', the task can just be ``go to library''. This will also help in easier end-user programming of robot.
	\item It provides a modular software design by decoupling tasks from the robot hardware.
	\item On-line adaptation of task/control approaches. That is, decision to execute force/acceleration control when approaching obstacle or corner etc.
\end{itemize}




%Based on mobility, the two main classes of robots are stationary and mobile. The different kinds of manipulators fall under stationary robots. The category of mobile robot includes wheeled, legged, flying, swimming robots, etc. Task specification and execution differs from each kind of robots. Considering the field of manipulators, they perform repetitive tasks such as pick and place, painting, welding etc, which involves continuous physical interaction with environment. 

%Additionally, \textit{safety} is one of the critical factors to be considered when designing robotic systems~\cite{shin2008hybrid}.



%The two main classes of robots are manipulators and mobile robots. Robot manipulators are extensively used in industrial environments. They are capable of performing repetitive tasks such as pick and place, painting, welding etc, that are specific to structured environment~\cite{craig2005introduction}. This is because of the their lack of mobility and limited reachability. Unlike manipulators, the mobile robots are not restricted to a confined workspace. Hence, they are employed in several applications involving unstructured environments such as logistics, inspection and maintenance tasks, defense sectors and many more.

% The robotic engineers and researchers from an extended period, have focused on the safety of robots and its workspace. The growing application of the two main classes of robots, i.e., manipulators and mobile robots in diverse fields adds to the necessity for safety. Additionally, growing robot capabilities due to the increasing industrial needs, the complexity of robot tasks have increased 

%The robotic tasks involve interaction with the environment either directly or indirectly. The robotic arms directly interact through the manipulation of objects in its workspace. However, the mobile robots interact indirectly,  by avoiding the obstacles while navigating from one point to another. In both cases, 

%The expanding applications of robots in diverse environments has increased the complexity of tasks. Implementing such task
%Robot manipulators are widely employed in an industrial environment. As manipulators are bulky and dangerous, the tasks are confined to a closed environment, away from humans. However, recently, the advancement in the field of manipulators has contributed to a safe interaction with humans. The increasing complexity in tasks has led to never-ending research in the collision handling systems. For instance, a robotic arm performing pick and place operation in a structured environment has to plan and execute the task safely by achieving dynamic collision avoidance.
%Additionally, there are various constraints imposed by the task specification. One such constraint would be to place the object vertically on the table without damaging the object and the workspace. Likewise, many such constraints are imposed as the complexity of tasks increases. There are software frameworks and dynamic solvers targeted to realize the constraints in real-time. \\
%
%Consequently, in the field of autonomous mobile robots, safe navigation is the crucial goal \cite{fox1997dynamic}. Due to their ability to navigate, mobile robots are often employed in applications such as logistics, security and defense, inspection and maintenance, cleaning, agriculture and many more. Typically navigating in populated environments, the mobile robot performs a task under changing external circumstances. Therefore, the robots must plan dynamically to respond to such unforeseen situations \cite{fox1997dynamic}. \\
%
%YET TO WRITE..............

%Robots are proliferating in various fields. The two main classes of robots are robot manipulators and mobile robots. Robot manipulators are used extensively in the industrial environment. The arm can perform repetitive tasks such as pick and place, welding, painting, etc. In spite of their popularity, they suffer from some disadvantages such as lack of mobility and limited reachability around its fixed base. Therefore their tasks are often simple and highly specific to a structured environment. The mobile robots were developed to overcome this restriction to a confined workspace. Due to their ability to move, mobile robots are employed in wide range of applications which include logistics, security and defense, inspection and maintenance, cleaning, agriculture and many more. The robot navigation has been implemented effectively by many approaches. However, these methods often perform obstacle avoidance. And the robot motions are controlled at velocity-level. In some circumstances, the objectives demand force/acceleration constraints if the robot is obliged to come in contact with the environment. The traditional velocity-based control cannot handle the constraints in force/acceleration level. Hence there is a necessity to manage these constraints in mobile robots. In contrast to mobile robots, the need for continuous physical interaction with the environment has already been recognized for several decades in the manipulators. This field is well researched in robotic arms that manipulate objects. The arm/joint parameters are bound by specific force constraints. Specifically, the end-effector joints are limited by allowable force on the object. For instance, consider a pick and place scenario, where the arm has to grasp a fragile glass and place it on a workbench. Here, the end-effector has to grip with a specific force such that the glass neither breaks nor slips out. Additionally, the task might impose multiple constraints, such as the end-effector must place the object perpendicular to the plane by applying limited forces. The arm must satisfy these dynamic constraints. The controller supervises these constraints at that instant of time. Besides, many such task constraints are imposed and hence dynamic solvers are used to realize them instantly. The Popov-Vereshchagin solver is one such dynamic solvers that can handle the requirements presented above in robotic manipulators. The Vereshchagin is a significant algorithm for the posture control of mobile manipulators and humanoid robots since such tasks typically require specifications of motion and/or force constraints on the end-effectors and other segments. The vereshchagin solver can be applied to closed as well as open kinematic chains. Since the wheels and base of the mobile robot can be modeled as a closed kinematic chain, the solver can be extended and applied to mobile robots.

%\subsection{...}





\section{Problem Statement} \label{sec:problem}
The methodologies proposed to implement mobile robot tasks, control the robot motions at velocity-level. At this level, the complex tasks that specifies position, force or acceleration constraints cannot be handled by the traditional approaches. The circumstances when robot comes across such constraints are explained with use cases below.

\begin{itemize}
	\item When the robot comes across an obstacle unexpectedly; \\ \textbf{Use case:} Consider an autonomous system navigating from point A to point B, and the robot comes across an obstacle unexpectedly. Using the conventional approaches, robot cannot execute instantaneous changes to the velocity. 
	\item When the robot task involves contact with an object; \begin{enumerate}
		\item \textbf{Use case:} Consider a multi-robot system performing logistic tasks in an industrial environment, where a robot has to physically latch itself to another through some means (e.g., a hook). For this purpose, the robot initially has to align and come in contact with the other robot physically to latch itself. This is an example of task imposing position and force constraint. 
		\item \textbf{Use case:} Consider a wall alignment problem. The usual procedure is to detect the wall, and the mounted sensors continuously compute the distance values from the wall. Based on these values, the robot adjusts its position. In spite of this traditional method, the project presents an approach to exploit the obstacle. Consider a virtual force is pulling the robot towards the wall, and at one point it comes in contact with the wall. This imposes acceleration constraint on robot motion.
	\end{enumerate}
%	The project addresses the safety constraints in the situations as explained in use cases. The project also addresses the issue regarding task specification for mobile bases. Generally, the manipulators involve a task specification strategy to fulfill the sequence of tasks. These tasks impose several constraints on the robot actions. Many software frameworks handle these constraints at the task level. However, in the field of mobile robots, there is no practical implementation of task specification approach. Below is a use case that depicts why task specification procedure would be helpful for mobile bases. 
	\item \textbf{Use case:} A mobile robot is performing logistic functions in a hospital environment. The task requires robot to carry objects to a destination simultaneously avoiding the robot entering a specified boundary. Since the robot is operating in a populated environment, it is required to maintain limited velocity and forces. In this use case, multiple constraints are imposed and are to be realized simultaneously. 
	
\end{itemize}

The primary definition of the problem statement is to introduce a modular and standardized framework to specify and control mobile robot tasks in the situations similar to use cases explained above. The aim of the project is, by using the dynamic model of the robot and by defining the natural and task constraints, the desired robot motions must be computed. 

%The above uses cases explains different types of constraints, that can be imposed by the task requirements. The velocity-controlled approaches cannot effectively handle these dynamic constraints. Hence the goal of the project is to extend and apply constrained hybrid dynamic solver, that realizes the task constraints by controlling the robot motions at position, force or acceleration level. Additionally, it provides an intuitive framework for task specification.

%
%\subsection{...}
%
%
%\subsection{...}
\end{document}