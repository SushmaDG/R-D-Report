%!TEX root = ../DevaramaniS-[RnD-MT]Report.tex

\chapter{Approach}

The following chapter describes the extension of the solver to mobile robot. In the previous chapter, the adaptation of the \textit{Vereshchagin solver} to the kinematic tree structure was presented. The initial step in applying the solver to mobile robots is to model the robot as a kinematic tree. 

A kinematic tree comprises of interconnected segments (links). A mobile robot can be modeled as kinematic tree with base of the robot as root of the tree and wheels as end-effectors. A typical kinematic tree description is provided by \hyperref[kdl]{KDL} library. 


