%!TEX root = ../DevaramaniS-[RnD-MT]Report.tex

\chapter{Extending the Vereshchagin hybrid dynamic solver to mobile robots}

As discussed in State of the art, there are various software frameworks and dynamic solvers employed specifically in manipulators to satisfy the constraints imposed by task requirements. These approaches considers the dynamic properties of the system and compute the desired motion. In case of mobile robots, there are approaches introduced to compute the control commands by considering the dynamics of the robot. However, there are no task specification frameworks or solvers that would provide a better procedure to handle instantaneous task specifications. Therefore, the main objective of the project is to 
extend and apply the \textit{Popov-Vereshchagin hybrid dynamic solver} to mobile robots. 


The main feature of the solver is that, given the task requirements, it calculates the instantaneous joint accelerations and control torques of a single end-effector. However, this can be extended to compute the desired motion of multiple end-effectors~\cite{shakhimardanov2015composable}. An autonomous mobile robot can be modeled as kinematic tree structure with wheels as end-effectors. Following sections describe extensions to tree structure and how the extended algorihtm can be applied to mobile robot.


\section{Extension to kinematic trees}

A theoretical description on extension of the solver to multiple end-effectors (kinematic tree structure), is presented by author Azamat Shakhimardanov in his dissertation~\cite{shakhimardanov2015composable}. The  conceptual and algorithmic explanation of this extension in each of the computational sweeps is presented below.

\subsection{Initial outward sweep}
The computations in outward sweep remains unchanged except the way of recursion. For a simple serial chain, the outward sweep is a single path from base to end-effector. This remains the same for kinematic tree structure as well, but the recursion must traverse to all the respective end-effectors~\cite{shakhimardanov2015composable}. 

In the Constrained hybrid dynamics algorithm (\ref{Algorithm1}), the initial outward sweep loops from $i=0$ (base) to $N-1$ (end-effector). The recursion must be changed to traverse from base to all the tree elements.

\subsection{Inward sweep}
The main modifications in inward sweep includes~\cite{shakhimardanov2015composable},
\begin{itemize}
	\item The apparent inertias ($H^a$) and forces ($F^a$) of parent segment must combine all the articulated inertias and forces computed by the child segments.
	\item The constraint force matrix ($A$), must combine the constraints applied at each of the end-effectors. Hence, the matrix will be expanded column-wise.
	\item The acceleration energy ($b_N$) is accumulated corresponding to the constraints arising from each of the end-effectors. 
	\item At the branching point, the constraints from each of the child segments must be joined into the acceleration constraint coupling matrix ($\mathcal{L}$). The matrix expands block-wise with the increase in number of constraints. Consider a three different dimensional constraint joined at the branching point, the resulting $\mathcal{L}$ matrix is~\cite{shakhimardanov2015composable},
	\begin{equation}
		\label{eq:L-matrix}
		\mathcal{L} = \begin{pmatrix}
		\mathcal{L}^k & 0 & 0\\
		0 & \mathcal{L}^l & 0\\
		0 & 0 & \mathcal{L}^m 
		\end{pmatrix}
	\end{equation} 
	In the above matrix, $k, l$ and $m$ are dimensions of constraints arising from each of the sub-chains. $\mathcal{L}^k, \mathcal{L}^l$ and $\mathcal{L}^m$ are constraint coupling matrices of respective sub-chains. 
\end{itemize}

\subsection{Resolving constraints at the base}

As explained in the section \ref{sec:computing-constraints}, the Gauss function, $\mathcal{Z}$ is minimized by applying the method of Lagrange multipliers. This results in constraint force magnitudes $\nu$ that corresponds to the acceleration constraints applied at the end-effector. But in case of tree structure, there are multiple end-effectors and corresponding constraint magnitudes must be computed. 


%The constraint calculation for a fixed base is given in the algorithm \ref{Algorithm1} in line \ref{21}. Here, the \textit{constraint force magnitudes} are calculated in a single step. However, in the case of kinematic tree, the constraint magnitudes corresponding to individual end-effectors must be calculated. 

{\color{red} citation here........}

\begin{equation}\label{eq:U}
	U_0^A = U_{bias, 0}^A + U_{tau, 0}^A + U_{qdd, 0}^A + U_{ext, 0}^A - b_N 
\end{equation}

\begin{equation}\label{eq:nu-float}
	\nu_{float} = \big[ (A_0^A)^T (H_0^A)^{-1} A_0^A - \mathcal{L}_0^A \big]^{-1} \big[ U_0^A - (A_0^A)^T (H_0^A)^{-1} F_{bias, 0}^A \big]
\end{equation}


The constraint calculation for a fixed base is given in the algorithm \ref{Algorithm1} (expression in line \ref{21}). For a floating base, $\nu$ cannot be directly calculated since the base acceleration is unknown. The constraint magnitudes are computed using the expression \ref{eq:nu-float}. Here, $A_0^A$ is a extended constraint force matrix for all the end-effector constraints expressed in root coordinates. $H_0^A$ is \textit{articulated-body inertia} denoted in Pl{\"u}ker coordinates. It is given by~\cite{featherstone2014rigid},

\begin{equation}
	\label{eq:RBI}
	H_0^A = \begin{pmatrix}
		I_0 & H_0 \\
		H_0^T & M_0
	\end{pmatrix}
\end{equation}

The notation $\mathcal{L}_0$ is acceleration coupling matrix expressed as \ref{eq:L-matrix} for given number of constraints. Further, the desired acceleration energy vector ($U_0^A$) expressed at root coordinates is given by \ref{eq:U}, where $b_N$ is the respective constraint acceleration energy vector. 


In case of kinematic chain structure, the base is fixed. Hence the root acceleration is equal to acceleration due to gravity (expression \ref{eq:acc}). 
\begin{equation}
	\label{eq:acc}
	\ddot{X}_0 = -\ddot{X}_g
\end{equation}

In case of a free floating base (for example: mobile robots, orbiting spacecraft) the base acceleration can be computed by~\cite{vereshchagin1989modeling}. 

\begin{equation}\label{eq:cartesian-acceleration}
\ddot{X}_{float, 0} = -(H_0^A)^{-1} (F_{bias, 0}^A + A_0^A  \nu_{float})
\end{equation}


\subsection{Final outward sweep}
The computations in final outward sweep remains the same, besides the recursion, which must traverse from root to all the end-effectors. The calculated $\nu_{float}$ is substituted in expression $\ref{23}$ and joint accelerations $(\ddot{q})$ are computed. Further in line $\ref{24}$, Cartesian acceleration is calculated.

\section{Conclusion}
The \textit{Vereshchagin solver} is exercised in robot manipulators to evaluate hybrid dynamics problem. \cite{KDLopensource} The algorithm has been implemented using \hyperref[kdl]{KDL} library (open-source), by Ruben Smits, Herman Bruyninckx, and Azamat Shakhimardanov~\cite{KDLopensource}. The \hyperref[kdl]{KDL} provides a framework to model and compute solutions to kinematics and dynamics problem. The real-time code that implements solver on manipulator is provided and maintained under \textit{Orocos (Open Robot Control Software) Project} and is written in C++ programming language. The approach that extend and apply the algorithm to mobile robots is described in the next chapter. 


%\section{Mobile Robots as Kinematic tree structure}
%
%For the approach presented in this project, the solver is extended to mobile robots. 
%
%
%\section{Implementation details}{........}
%
%\begin{table}[h]
%	
%	\renewcommand{\arraystretch}{2.5}
%	\resizebox{\textwidth}{!}{%
%		\begin{tabular}{| l | l | l | l | l |}
%			\hline
%			\multicolumn{2}{| M{7.5cm}|}{\textbf{Inputs}}    & \multicolumn{3}{M{15cm}|}{\textbf{Outputs}} \\ \hline
%			
%			\textit{\textbf{External force ($F_{ext}$)}} & \textit{\textbf{Joint Torques($\tau$)}} & \textit{\textbf{Joint accelerations ($\Ddot{q}$)}} & \textit{\textbf{Root acceleration}} & \textit{\textbf{Control torques ($\tau_{control}$)}} \\ \hline
%			
%			\begin{tabular}[c]{@{}l@{}}$\gamma = 10$ \\ $[0, 0, 0, 0,  0, 10]$ \end{tabular}   &  $[0, 0, 0, 0, 0, 0, 0, 0]$ &  \begin{tabular}[c]{@{}l@{}} $[-0.651743,   -1.36504, -3.97664,  -1.36552,$ \\ $-3.97664, -1.36552, -0.651743, -1.36504]$\end{tabular}   & \begin{tabular}[c]{@{}l@{}} $\gamma = 1.4962 $ \\  $x = -0.0368095$ \end{tabular} & 	\begin{tabular}[c]{@{}l@{}} $[0.303226, -0, -0.126057, -0,$ \\$ -0.126057, -0, 0.303226, 0]$ \end{tabular} \\ \hline
%			
%			\begin{tabular}[c]{@{}l@{}}$x = 50$ \\ $[50, 0, 0, 0,  0, 0]$ \end{tabular}   &  $[0, 0, 0, 0, 0, 0, 0, 0]$ &  \begin{tabular}[c]{@{}l@{}} $[3.49264, 0.168407, 3.90164    0.168467,$ \\ $3.90164, 0.168467, 3.49264, 0.168407]$\end{tabular}   & \begin{tabular}[c]{@{}l@{}} $\gamma = -0.184047 $ \\  $x = 0.158089$ \end{tabular} & 	\begin{tabular}[c]{@{}l@{}} $[0.243069, -0, 0.238214, -0 ,$ \\ $0.238214, -0, 0.243069, -0]$ \end{tabular} \\ \hline
%			
%			$[0, 0, 0, 0,  0, 0]$  &  \begin{tabular}[c]{@{}l@{}} $\tau_{d_1} = 1.0$ \\ $\tau_{d_2} = 1.0$ \\ $\tau_{d_3} = 1.0$ \\ $\tau_{d_4} = 1.0$  \end{tabular} &  \begin{tabular}[c]{@{}l@{}} $[ 0.329141, 34.0727, 0.549795, 34.0727,$ \\  $0.757357, 34.1083, 0.12158, 34.1082]$\end{tabular}   & \begin{tabular}[c]{@{}l@{}}  $x = 0.0111047$ \\ $y = 0.00467013$ \\ $\gamma = -0.192697$ \end{tabular} & 	\begin{tabular}[c]{@{}l@{}} $[0.0695287, -0, -0.0665266, -0,$ \\ $-0.0539123, -0, 0.0463929, -0]$ \end{tabular} \\ \hline
%			
%			
%			
%		\end{tabular}
%	}\renewcommand{\arraystretch}{3}
%	\caption{Experimental analysis}
%	\label{Decomposition Techniques}
%	
%\end{table}
