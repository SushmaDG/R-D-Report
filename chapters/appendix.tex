%!TEX root = ../DevaramaniS-[RnD-MT]Report.tex
\chapter{Dynamic equation of motion}\label{chap:dynamic}
The general dynamic equation of motion of a rigid body is expressed as \cite{featherstone2014rigid} \cite{shakhimardanov2015composable}, 
\begin{equation}
	\label{eq:dynamic}
	M(q)\ddot{q} + C(q, \dot{q}) = \tau 
\end{equation}

where, $M(q)$ represents mapping from motion domain($M^n$) to force domain($F^n$). $C(q, \dot{q})$ is the Coriolis and Centrifugal forces acting on the rigid body. Both these quantities are dependent on $q$, $\dot{q}$, $\ddot{q}$ and the physical model of rigid body \cite{featherstone2014rigid}

The dynamics problem is divided into forward and inverse dynamics. Computing the acceleration($\ddot{q}$), given the input forces($\tau$) is termed as \textit{forward dynamics} problem. Conversely, \textit{Inverse dynamics} problem calculates the forces, $\tau$ given accelerations $\ddot{q}$.

The rigid body is generally subjected to various motion constraints that changes the form of the dynamics equation. The extended equation is given by \cite{shakhimardanov2015composable},

\begin{equation}
\label{eq:extendeddynamic}
M(q)\ddot{q} = \tau_a(q) - \tau_c(q) -  C(q, \dot{q})
\end{equation}

In the above equation, $\tau_a$ represents input forces and $\tau_c$ is the constraint forces from the task specification.

\chapter{Pl{\"u}cker Notations for Spatial vectors}\label{chap:plucker}
Your second chapter appendix

\chapter{Pl{\"u}cker Notations for Spatial cross products}\label{chap:cross}

There are two spatial cross product operators expressed using Pl{\"u}cker notations are, $\times$ and $\times^*$ \cite{featherstone2014rigid}. The operators can be regarded as dual to each other. The matrix representation of these operators are deduced as \cite{featherstone2014rigid},

	\begin{equation}
	\label{eq:cross1}
		\hat v_O \times = \begin{bmatrix}
			\omega \\ v_O
		\end{bmatrix} \times = \begin{bmatrix}
			\omega \times & 0 \\
			v_O \times & \omega \times
		\end{bmatrix}
	\end{equation}
	
	and,
	
	\begin{equation}
	\label{eq:cross2}
	\hat v_O \times^* = \begin{bmatrix}
	\omega \\ v_O
	\end{bmatrix} \times^* = \begin{bmatrix}
	\omega \times & v_O \times \\
	0 & \omega \times
	\end{bmatrix}
	\end{equation}
	
	